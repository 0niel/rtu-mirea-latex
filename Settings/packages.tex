\usepackage{polyglossia} % языковой пакет

\usepackage{pdfpages} % пакет для импорта pdf-файлов

\usepackage[labelsep=period]{caption}

\usepackage{caption}

\usepackage{graphicx} % пакет для использования графики (чтобы вставлять рисунки, фотографии и пр.)

\usepackage{amsmath} % поддержка математических символов

\usepackage{url} % поддержка url-ссылок

\usepackage{natbib} % менеджер цитирования natlib.

\bibliographystyle{unsrtnat} % выбираем стиль библиографии отсюда: https://www.overleaf.com/learn/latex/Natbib_bibliography_styles

\setcitestyle{authoryear, open={(},close={)}} % Определяем стиль цитирования. Указываем, чтобы цитирование в тексте вставлялось в формате (Автор, год). 

\usepackage{multirow} % таблицы с объединенными строками

\usepackage{hyperref} % пакет для интеграции гиперссылок

\usepackage{indentfirst} % пакет для отступа абзаца


\usepackage{chngcntr} % пакет подписей и нумерации рисунков



%%%%%%%%%%%%%%%%%%%%%%%%%%%%%%%%%%%%%%%%%%%%%%%%%%%%%%%%%%%%%%